\documentclass[a4paper]{article}

\usepackage[portuguese,english]{babel}
\usepackage[utf8]{inputenc}
\usepackage[T1]{fontenc}

\newcommand{\documentTitle}{Data Analysis and Transformation \\ TP3: Sampling Theorem, Discrete Fourier Transform and Audio and Image Signals' Filtering}
\newcommand{\pdfTitle}{[ATD] TP3: Sampling Theorem, Discrete Fourier Transform and Audio and Image Signals' Filtering}
\newcommand{\documentAuthors}{José Ribeiro (2008112181, jbaia@student.dei.uc.pt) \\ Pedro Magalhães (2009117002, pjrosa@student.dei.uc.pt)}

\title{\documentTitle}
\author{\documentAuthors}

\usepackage{hyperref}
\hypersetup{
	pdftitle = \pdfTitle
	,pdfauthor = \documentAuthors
	,pdfsubject = {Data Analysis and Transformation}
	,pdfkeywords = {Data Analysis and Transformation} {Sampling Theorem} {Discrete Fourier Transform}
	,pdfborder = {0 0 0}
}

%\usepackage{subfig}
%\usepackage{amsmath}
%\usepackage{array}
\usepackage{anysize}
\usepackage{lscape}
\usepackage{amsmath}
\usepackage{graphicx}
\usepackage{caption}
\usepackage{amssymb}
%\usepackage[pdftex]{graphicx}
%\usepackage[table]{xcolor}

\hyphenation{}

\marginsize{2.3cm}{2.3cm}{3cm}{3cm}

\makeatletter

\begin{document}
\maketitle
\cleardoublepage

\tableofcontents
\cleardoublepage

\setlength{\parindent}{1cm}
\setlength{\parskip}{0.3cm}

\section{Exercise 1}
\subsection{Exercise 1.1}
\label{subsec:ex_1_1}
\noindent Após a substituição do número de grupo na fórmula, procedeu-se à sua simplificação segundo a forma de Série de Fourier.
\begin{eqnarray}
	x(t) & = & -1 + 3 \, sin(30 \, \pi \, t) + 4 \, sin\left( 12 \, \pi \, t - \frac{\pi}{4} \right) \, cos(21 \, \pi \, t) \\
	& = & cos(\pi) + 3 \, cos\left( 30 \, \pi \, t - \frac{\pi}{2} \right) + 2 \, \left( sin\left( \left( 12 \, \pi \, t - \frac{\pi}{4} \right) + 21 \, \pi \, t \right) + sin\left( \left( 12 \, \pi \, t - \frac{\pi}{4} \right) - 21 \, \pi \, t\right) \right) \\
	& = & cos(\pi) + 3 \, cos\left( 30 \, \pi \, t - \frac{\pi}{2} \right) + 2 \, sin\left( 33 \, \pi \, t - \frac{\pi}{4} \right) + 2 \, sin\left( -9 \, \pi \, t - \frac{\pi}{4} \right) \\
	& = & cos(\pi) + 3 \, cos\left( 30 \, \pi \, t - \frac{\pi}{2} \right) + 2 \, cos\left( 33 \, \pi \, t - \frac{\pi}{4} - \frac{\pi}{2} \right) + 2 \, cos\left( -9 \, \pi \, t - \frac{\pi}{4} - \frac{\pi}{2} \right) \\
	& = & cos(\pi) + 3 \, cos\left( 30 \, \pi \, t - \frac{\pi}{2} \right) + 2 \, cos\left( 33 \, \pi \, t - \frac{3 \, \pi}{4} \right) + 2 \, cos\left( -9 \, \pi \, t - \frac{3 \, \pi}{4} \right) \\
	& = & 1 \, cos(0 \, t + \pi) + 2 \, cos\left( 9 \, \pi \, t + \frac{3 \, \pi}{4} \right) + 3 \, cos\left( 30 \, \pi \, t + \frac{3 \, \pi}{2} \right) + 2 \, cos\left( 33 \, \pi \, t + \frac{5 \, \pi}{4} \right)
\end{eqnarray}

\noindent Com a fórmula nesta forma tornam-se evidentes a frequência fundamental e a frequência máxima. A frequência angular fundamental ($\omega_0$) é dada pelo máximo divisor comum entre as várias frequências angulares, i. e.,
\begin{equation}
	\omega_0 = mdc \, (0, 9 \, \pi, 30 \, \pi, 33 \, \pi) = 3 \, \pi
\end{equation}

\noindent A frequência angular máxima ($\omega_{max}$) é a máxima de todas as frequências angulares, pelo que
\begin{equation}
	\omega_{max} = max \, (0, 9 \, \pi, 30 \, \pi, 33 \, \pi) = 33 \, \pi
\end{equation}

\noindent Segundo o Teorema de Nyquist-Shannon, para que um sinal possa ser reconstruído sem erro a partir de uma sequência resultante da sua amostragem, a frequência de amostragem utilizada deverá ser superior ao dobro da maior frequência presente no sinal original, ou seja,
\begin{equation}
	f_s > 2 \, f_{max}
\end{equation}

\noindent De forma a garantir que a frequência de amostragem escolhida obedece à condição e é um múltiplo da frequência fundamental,
\begin{equation}
	f_s = 2 \, f_{max} + n \, f_0, ~ n \in \mathbb{N}
\end{equation}

\noindent dado que $f_{max}$ é múltiplo de $f_0$. Aplicando esta fórmula a frequências angulares,
\begin{equation}
	w_s = 2 \, w_{max} + n \, w_0, ~ n \in \mathbb{N}
\end{equation}

\noindent Escolheu-se $n = 2$ dado que, como $w_0$ e $w_{max}$ são múltiplos de $\pi$, $w_s$ será múltipla de $2 \, \pi$, o que garante que a sua conversão de $radianos$ para $hertz$ dará um número inteiro. Conclui-se que
\begin{equation}
	w_s = 2 * 33 \, \pi + 2 * 3 \, \pi = 72 \, \pi \, rad/s
\end{equation}

\noindent e consequentemente
\begin{equation}
	f_s = \frac{w_s}{2 \, \pi} = \frac{72 \, \pi}{2 \, \pi} = 36 \, Hz
\end{equation}

\begin{equation}
	T_s = \frac{1}{f_s} = \frac{1}{36} \, s
\end{equation}

\noindent A expressão de $x[n]$ é, consequentemente,
\begin{eqnarray}
	x[n] = x(t)|_{t = n \, T_s} = & cos(\pi) + 2 \, cos\left( 9 \, \pi \, n \, T_s + \frac{3 \, \pi}{4} \right) + 3 \, cos\left( 30 \, \pi \, n \, T_s + \frac{3 \, \pi}{2} \right) + 2 \, cos\left( 33 \, \pi \, n \, T_s + \frac{5 \, \pi}{4} \right) \\
	= & cos(\pi) + 2 \, cos\left( \frac{9 \, \pi}{36} \, n + \frac{3 \, \pi}{4} \right) + 3 \, cos\left( \frac{30 \, \pi}{36} \, n + \frac{3 \, \pi}{2} \right) + 2 \, cos\left( \frac{33 \, \pi}{36} \, n + \frac{5 \, \pi}{4} \right) \\
	= & cos(\pi) + 2 \, cos\left( \frac{\pi}{4} \, n + \frac{3 \, \pi}{4} \right) + 3 \, cos\left( \frac{5 \, \pi}{6} \, n + \frac{3 \, \pi}{2} \right) + 2 \, cos\left( \frac{11 \, \pi}{12} \, n + \frac{5 \, \pi}{4} \right)
\end{eqnarray}

\clearpage
\section{Attachments}
\subsection{Trigonometric identities}
\label{subsec:trigident}
\noindent As simplicações da alínea \emph{\nameref{subsec:ex_1_1}} faz uso das seguintes igualdades:
\begin{equation}
	sin(\theta) \, cos(\varphi) = \frac{sin(\theta - \varphi) + sin(\theta + \varphi)}{2}
\end{equation}

\begin{equation}
	cos(-\theta) = cos(\theta)
\end{equation}

\end{document}
