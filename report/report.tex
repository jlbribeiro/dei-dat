\documentclass[a4paper]{article}

\usepackage[portuguese,english]{babel}
\usepackage[utf8]{inputenc}
\usepackage[T1]{fontenc}

\newcommand{\documentTitle}{Análise e Transformação de Dados \\ TP1: Fundamentos de Sinais e Sistemas}
\newcommand{\pdfTitle}{[ATD] TP1: Fundamentos de Sinais e Sistemas}
\newcommand{\documentAuthors}{José Ribeiro (2008112181, jbaia@student.dei.uc.pt) \\ Pedro Magalhães (2009XXXXXX, XXXXXXX@student.dei.uc.pt)}

\title{\documentTitle}
\author{\documentAuthors}

\usepackage{hyperref}
\hypersetup{
	pdftitle = \pdfTitle
	,pdfauthor = \documentAuthors
	,pdfsubject = {Projecto DG/AAC 2012}
	,pdfkeywords = {DG/AAC} {Pelouro da Cultura} {Projecto} {SMS Gateway}
	,pdfborder = {0 0 0}
}

%\usepackage{subfig}
%\usepackage{amsmath}
%\usepackage{array}
\usepackage{anysize}
\usepackage{lscape}
%\usepackage[pdftex]{graphicx}
%\usepackage[table]{xcolor}

\hyphenation{}

\marginsize{3.5cm}{3.5cm}{3cm}{3cm}

\makeatletter

\begin{document}
\maketitle
\cleardoublepage

\tableofcontents
\cleardoublepage

\setlength{\parindent}{1cm}
\setlength{\parskip}{0.3cm}

\section{Exercício 1}
\subsection{Exercício 1.1}
\indent \indent 

% ...


\section{Exercício 2}
\subsection{Exercício 2.1}
\indent \indent 

% ...


\section{Exercício 3}
\subsection{Exercício 3.1}
\indent \indent 

% ...


\section{Exercício 4}
\subsection{Exercício 4.1}
\indent \indent 

% ...


\section{Exercício 5}
\subsection{Exercício 5.1}
\indent \indent 

% ...


\section{Exercício 6}
\subsection{Exercício 6.1}
\indent \indent 

% ...

\end{document}
