\documentclass[a4paper]{article}

\usepackage[portuguese,english]{babel}
\usepackage[utf8]{inputenc}
\usepackage[T1]{fontenc}

\newcommand{\documentTitle}{Data Analysis and Transformation \\ TP1: Fundamentals of Signals and Systems}
\newcommand{\pdfTitle}{[ATD] TP1: Fundamentals of Signals and Systems}
\newcommand{\documentAuthors}{José Ribeiro (2008112181, jbaia@student.dei.uc.pt) \\ Pedro Magalhães (2009117002, pjrosa@student.dei.uc.pt)}

\title{\documentTitle}
\author{\documentAuthors}

\usepackage{hyperref}
\hypersetup{
	pdftitle = \pdfTitle
	,pdfauthor = \documentAuthors
	,pdfsubject = {Data Analysis and Transformation}
	,pdfkeywords = {Data Analysis and Transformation} {Signal Processing}
	,pdfborder = {0 0 0}
}

%\usepackage{subfig}
%\usepackage{amsmath}
%\usepackage{array}
\usepackage{anysize}
\usepackage{lscape}
\usepackage{amsmath}
\usepackage{graphicx}
\usepackage{caption}
\usepackage{amssymb}
%\usepackage[pdftex]{graphicx}
%\usepackage[table]{xcolor}

\hyphenation{}

\marginsize{2.7cm}{2.7cm}{3cm}{3cm}

\makeatletter

\begin{document}
\maketitle
\cleardoublepage

\tableofcontents
\cleardoublepage

\setlength{\parindent}{1cm}
\setlength{\parskip}{0.3cm}

\section{Exercise 1}
\subsection{Exercise 1.1}
\noindent Após substituição das variáveis pelo número de grupo, obtém-se a seguinte expressão:
\begin{equation}
	y[n] = 0.3137 \, x[n - 3] - 0.1537 \, x[n - 5] + 2.3 \, y[n - 1] - 1.74 \, y[n - 2] + 0.432 \, y[n - 3] \\
\end{equation}

\noindent A partir da expressão obtém-se
\begin{equation}
	y[n] - 2.3 \, y[n - 1] + 1.74 \, y[n - 2] - 0.432 \, y[n - 3] = 0.3137 \, x[n - 3] - 0.1537 \, x[n - 5] \\
\end{equation}

\noindent Aplicada a Transformada Z em ambos os lados da expressão

\begin{equation}
	\mathcal{Z}(y[n] - 2.3 \, y[n - 1] + 1.74 \, y[n - 2] - 0.432 \, y[n - 3]) = \mathcal{Z}(0.3137 \, x[n - 3] - 0.1537 \, x[n - 5])
\end{equation}

\noindent é possível obter-se a Função de Transferência do Sistema na forma polinomial.

\noindent Dado que
\begin{eqnarray}
	G(z) & = & H(z)|_{\text{condições iniciais nulas}} \\
		 & = & \mathcal{Z}\{h[n]\} \\
		 & = & \frac{Y(z)}{X(z)}
\end{eqnarray}

\noindent tem-se que
\begin{eqnarray}
	&\mathcal{Z}\{y[n] - 2.3 \, y[n - 1] + 1.74 \, y[n - 2] - 0.432 \, y[n - 3]\} = \mathcal{Z}\{0.3137 \, x[n - 3] - 0.1537 \, x[n - 5]\} \\
	&Y(z) - 2.3 \, z^{-1} \, Y(z) + 1.74 \, z^{-2} \, Y(z) - 0.432 \, z^{-3} \, Y(z) = 0.3137 \, z^{-3} \, X(z) - 0.1537 \, z^{-5} \, X(z) \\
	&Y(z) (1 - 2.3 \, z^{-1} + 1.74 \, z^{-2} - 0.432 \, z^{-3}) = X(z)(0.3137 \, z^{-3} - 0.1537 \, z^{-5}) \\
	&\frac{Y(z)}{X(z)} = \frac{0.3137 \, z^{-3} - 0.1537 \, z^{-5}}{1 - 2.3 \, z^{-1} + 1.74 \, z^{-2} - 0.432 \, z^{-3}}
\end{eqnarray}

\noindent pelo que se obtém que
\begin{equation}
	G(z) = \frac{0.3137 \, z^{-3} - 0.1537 \, z^{-5}}{1 - 2.3 \, z^{-1} + 1.74 \, z^{-2} - 0.432 \, z^{-3}}
\end{equation}

\subsection{Exercise 1.2}
\subsubsection{Exercise 1.2.1}
TODO

% \begin{equation}
%	x_{1}(t) = 2 \, sin(2 t) \, cos(11 t) + 5 \, cos^2 \, (8t)
%	\label{eq:origsubst}
% \end{equation}

% \begin{eqnarray}
% \label{eq:simplification}
% x_{1}(t) & = & 2 \, sin(2 t) \, cos(11 t) + 5 \, cos^2 \, (8t) \\
% 	 & = & sin(2 t + 11 t) + sin(2 t - 11 t) + 5 \, cos^2 \, (8t) \\
% 	 & = & sin(13 t) + sin(- 9 t) + 5 \, cos^2 \, (8t) \\
% 	 & = & - cos\left(13 t + \frac{\pi}{2}\right) - cos\left(- 9 t + \frac{\pi}{2}\right) + 5 \, cos^2 \, (8t) \\
% 	 & = & cos\left(13 t + \frac{3 \pi}{2}\right) + cos\left(- 9 t + \frac{3 \pi}{2}\right) + 5 \, cos^2 \, (8t) \\
% 	 & = & cos\left(13 t + \frac{3 \pi}{2}\right) + cos\left(9 t + \frac{\pi}{2}\right) + 5 \, cos^2 \, (8t) \\
% 	 & = & cos\left(13 t + \frac{3 \pi}{2}\right) + cos\left(9 t + \frac{\pi}{2}\right) + 5 \left(\frac{1 + cos(2 * 8 t)}{2}\right) \\
% 	 & = & cos\left(13 t + \frac{3 \pi}{2}\right) + cos\left(9 t + \frac{\pi}{2}\right) + \frac{5}{2} \, cos(0) + \frac{5}{2} \, cos(16 t) \\
% 	 & = & \frac{5}{2} \, cos(0) + cos\left(9 t + \frac{\pi}{2}\right) + cos\left(13 t + \frac{3 \pi}{2}\right) + \frac{5}{2} \, cos(16 t)
% \end{eqnarray}

\end{document}
